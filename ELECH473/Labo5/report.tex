\documentclass[a4paper,10pt]{article}
\usepackage[utf8]{inputenc}
\usepackage[T1]{fontenc}
\usepackage[english]{babel}
\usepackage[a4paper, margin=1.0in]{geometry}
\usepackage{graphicx}

\title{ELEC-H-473 : Microprocessor Architectures\\ Lab 5 : dsPIC33 1/2}
\author{Arnaud Rosette, Simon Picard}

\begin{document}

\begin{center}
\textbf{ELEC-H-473 : Microprocessor Architectures\\ Lab 5 : dsPIC33 1/2\\Arnaud Rosette, Simon Picard}
\end{center}

\subsubsection*{Question 1}
It is a Modified Harvard architecture with 16-bit wide data path and 24-bit wide instructions.\\
There are 83 base instructions (mostly 1 word/1 cycle).\\
The number of register is sixteen and these registers are 16-bit General Purpose Registers. There are also two 40-bit accumulators. The addressing modes are immediate, direct and indirect.

%\begin{itemize}
%>\item Modified Harvard architecture
%\item C compiler optimized instruction set
%>\item 16-bit wide data path
%>\item 24-bit wide instructions
%\item Linear program memory addressing up to 4M
%instruction words
%\item Linear data memory addressing up to 64 Kbytes
%>\item 83 base instructions: mostly 1 word/1 cycle
%>\item Sixteen 16-bit General Purpose Registers
%>\item Two 40-bit accumulators: With rounding and saturation options
%>\item Flexible and powerful addressing modes: Indirect, Modulo and Bit-Reversed
%\item Software stack
%\item 16 x 16 fractional/integer multiply operations
%\item 32/16 and 16/16 divide operations
%\item Single-cycle multiply and accumulate: Accumulator write back for DSP operations and Dual data fetch
%\item Up to +-16-bit shifts for up to 40-bit data
%\end{itemize}

\subsubsection*{Question 2}
A microprocessor contains only a CPU while the microcontronller contains a CPU, some RAMn some ROM and other peripherals all embedded on a single chip.\\
A microcontronller is a standalone component which is not the case for a microprocessor.
\subsubsection*{Question 3}
The addressing modes for data are : file register, register direct, register indirect and immediate.

\subsubsection*{Question 4}
\begin{itemize}
\item \textbf{File register addressing} : MOV 0x27FE, W0 (move data stored at 0x27FE to W0)
\item \textbf{Register direct addressing} : EXCH W2,W3 (exchange the content of registers W2 and W3)
\item \textbf{Register indirect addressing} : MOV [W0],[W13] (move data stored at address in W0 to address in W13)
\item \textbf{Immediate addressing} : ADD.B $\#$0x10,W0 (add 0x10 to W0)
\end{itemize}

\subsubsection*{Question 5}
The size of an instruction is 24 bits.\\
Yes.//il faut expliquer

\subsubsection*{Question 6}

\subsubsection*{Question 7}
This header file provides a C interface for I/O to programmers. It makes real registers usable from C code.

\subsubsection*{Question 8}
The volatile keyword is used in this header file because some of the variables in this file are real register. So the value of these variables could change unexpectedly (the value can be changed by an interrupt service routine). The volatile keyword tells the compiler to do not make optimizations.

\subsubsection*{Question 9}
\begin{itemize}
 \item \textbf{char, signed char} : -128 to 127
 \item \textbf{unsigned char} : 0 to 255
 \item \textbf{short, signed short} : -32768 to 32767
 \item \textbf{unsigned short} : 0 to 65535
 \item \textbf{int, signed int} : -32768 to 32767
 \item \textbf{unsigned int} : 0 to 65535
 \item \textbf{long, signed long} : -$2^{31}$ to $2^{31}$-1
 \item \textbf{unsigned long} : 0 to $2^{32}$-1
 \item \textbf{long long**, signed long long**} : -$2^{63}$ to $2^{63}$-1
 \item \textbf{unsigned long long**} : 0 to $2^{64}$-1
 \item \textbf{float} : $-3,40282326*10^{38}$ to $3,40282326*10^{38}$ (normalized)
 \item \textbf{double*} : $-3,40282326*10^{38}$ to $3,40282326*10^{38}$ (normalized)
 \item \textbf{long double} : $-1,7976931*10^{308}$ to $1,7976931*10^{308}$ (normalized)
\end{itemize}

\subsubsection*{Question 10}
The integer types are signed by default.

\subsubsection*{Question 11}
There is no assumption anymore in the fact that integers are signed or not by default. The programmer has to specify if he wants to use a signed or an unsigned integer. Furthermore, the number of bits used to represent a type is specified in the name of the type. It makes the code less dependent of the processor.

\subsubsection*{Question 12}
No answer needed.

\subsubsection*{Question 13}
The variables are stored in the stack.

\subsubsection*{Question 14}
The variables are all initialized by moving an immediate value in one or several registers and then the value of  this (or these) register(s) is stored in the memory. The types contained in 8 bits use two mov.b instructions in order to initialize a variable, those which are contained in 16 bits use two mov.w instructions, those which are contained in 32 bits use four mov.w instructions and those which are contained in 64 bits use eight mov.w instructions. The indirect addressing mode is used here, a register (here w14) contains a base address and then an offset is used in order to store variables at different locations in the memory.

\subsubsection*{Question 15}
\begin{verbatim}
INT8U a = 180; // 45 cycle
INT8S b = -105; // 47 cycle

INT16U c = 0x3210; // 49 cycle
INT16S d = -30000; // 51 cycle

INT32U e = 129856; // 55 cycle
INT32S f = -128423; // 59 cycle

INT64U g = 0xFEDCBA9876543210; // 67 cycle
INT64S h = -0x123456789ABCDF0; // 75 cycle

FP32 i = 1.27E+33; // 79 cycle
FP64 j = 8.76543E-18; // 87 cycle

glob1=132; // 88 cycle
\end{verbatim}

\subsubsection*{Question 16}
All the variable but glob1 are load at the first step of the debugger, variable g and h have the same hexadecimal value while not the same decimal value.\\
glob1 is set once the debugger execute its line.

\subsubsection*{Question 17}
The value are temporarily stored in registers from WREG4 to WREG7, for INT16U it use one register, 2 for INT32U and 4 for INT64U. The higher the number of the register the higher the importance in the number, i.e, WREG7 is the MSB in a INT64U, so to read a INT64U in reegister you have to concat WREG7, WREG6, WREG5, WREG4.
In the register you can't detect signed number directly.

\subsubsection*{Question 18}
The warnings are about unused variable and non explicit casts, nothing to be worried about.//si, il y a un warning concernant l'assignation de la variable globale qui est important parce que 1) il y a un cast de int vers adresse et 2) assigner une adresse comme ca n'est jamais conseillé.

\subsubsection*{Question 19}
This global variable is defined line 4, out of the main function.\\
It is initialised line 23, at the end of the main function.

\subsubsection*{Question 20}
? TODO//commmenter la ligne de declaration

\subsubsection*{Question 21}
First problem : too big integer gor unsigned integer, therefore truncated.\\
Too big integer for signed integer.\\
Too big double assigned to infinity.\\
Too small double assigned to 0.\\
The negative value are stored using the ones' complement.

\subsubsection*{Question 22}
? TODO//addressage direct

\subsubsection*{Question 23}
Firstly, the value of the variable are load in the register, if size of the variable is smaller than the size of the value then the variable will be assigned with the LSB of the value.\\
\end{document}
