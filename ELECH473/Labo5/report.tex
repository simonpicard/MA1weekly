\documentclass[a4paper,10pt]{article}
\usepackage[utf8]{inputenc}
\usepackage[T1]{fontenc}
\usepackage[english]{babel}
\usepackage[a4paper,top=2.5cm, bottom=2.5cm, left=2.5cm, right=2.5cm]{geometry}
\usepackage{graphicx}

\title{ELEC-H-473 : Microprocessor Architectures\\ Lab 5 : dsPIC33 1/2}
\author{Arnaud Rosette, Simon Picard}

\begin{document}
\maketitle
\section*{Question 1}

\section*{Question 2}

\section*{Question 3}
The addressing modes for data are : file register, register direct, register indirect and immediate.

\section*{Question 4}
\subsection*{File register addressing}
MOV 0x27FE, W0 (move data stored at 0x27FE to W0)
\subsection*{Register direct addressing}
EXCH W2,W3 (exchange the content of registers W2 and W3)
\subsection*{Register indirect addressing}
MOV [W0],[W13] (move data stored at address in W0 to address in W13)
\subsection*{Immediate addressing}
ADD.B $\#$0x10,W0 (add 0x10 to W0)

\section*{Question 5}
The size of an instruction is 24 bits. %TODO: 2eme sous question

\section*{Question 6}

\section*{Question 7}
This header file provides a C interface for I/O to programmers. It makes real registers usable from C code.

\section*{Question 8}
The volatile keyword is used in this header file because some of the variables in this file are real register. So the value of these variables could change unexpectedly (the value can be changed by an interrupt service routine). The volatile keyword tells the compiler to do not make optimizations.

\section*{Question 9}
\begin{itemize}
 \item \textbf{char, signed char} : -128 to 127
 \item \textbf{unsigned char} : 0 to 255
 \item \textbf{short, signed short} : -32768 to 32767
 \item \textbf{unsigned short} : 0 to 65535
 \item \textbf{int, signed int} : -32768 to 32767
 \item \textbf{unsigned int} : 0 to 65535
 \item \textbf{long, signed long} : -$2^{31}$ to $2^{31}$-1
 \item \textbf{unsigned long} : 0 to $2^{32}$-1
 \item \textbf{long long**, signed long long**} : -$2^{63}$ to $2^{63}$-1
 \item \textbf{unsigned long long**} : 0 to $2^{64}$-1
 \item \textbf{float} :
 \item \textbf{double*} :
 \item \textbf{long double} :
\end{itemize}

\section*{Question 10}
The integer types are signed by default.

\section*{Question 11}
There is no assumption anymore in the fact that integers are signed or not by default. The programmer has to specify if he wants to use a signed or an unsigned integer. Furthermore, the number of bits used to represent a type is specified in the name of the type. It makes the code less dependent of the processor.

\section*{Question 12}
No answer needed.

\section*{Question 13}
The variables are stored in the stack.

\section*{Question 14}
The variables are all initialized by moving an immediate value in one or several registers and then the value of  this (or these) register(s) is stored in the memory. The types contained in 8 bits use two mov.b instructions in order to initialize a variable, those which are contained in 16 bits use two mov.w instructions, those which are contained in 32 bits use four mov.w instructions and those which are contained in 64 bits use eight mov.w instructions.

\section*{Question 15}
\end{document}
